\documentclass{article}

\usepackage[utf8]{inputenc}
\usepackage{caption}
\usepackage{subcaption}
\usepackage{natbib}
\usepackage{soul}
\usepackage{graphicx}
\usepackage{listings}
\usepackage[hyphens]{url}
\usepackage[table,xcdraw]{xcolor}
\usepackage{color}
\usepackage{textpos}
\usepackage{glossaries}
\usepackage{hyperref}
\usepackage{acronym}
\usepackage{verbatim}
\usepackage{amsmath}
\usepackage{amssymb}
\usepackage{comment}


\newcommand{\emailaddr}[1]{\href{mailto:#1}{\texttt{#1}}}

\begin{document}
\begin{titlepage}

  \newcommand{\HRule}{\rule{\linewidth}{0.5mm}}
  \center
  
  \textsc{\Large Department of Computer Science And Engineering}\\[0.5cm]
  
  \textsc{\Large University of Bologna}\\[0.6cm]
  
  \hrule width \hsize \kern 1mm \hrule width \hsize height 2pt 
  \vspace{0.8cm}
  { \large \bfseries DealerPro}\\[0.6cm]
  { \large \bfseries Descrizione dell'approccio utilizzato}\\[0.6cm]
  { \large Project Management}\\[0.6cm]
  
  
  {\bfseries{June, 2023}
  \hfill
  \bfseries{Davide Domini}}\\[0.6cm]
  
  \hrule width \hsize height 2pt \kern 1mm \hrule width \hsize height 1pt
  \vspace{0.4cm}
  %\begin{abstract}
  %\end{abstract}
  
  \end{titlepage}
  %%GLOSSARIO https://it.overleaf.com/project/646f34c78c7d27e771d0078f
  \clearpage

  \tableofcontents

  \newpage
  \section{Scoping/Initiation}

  La nostra software house è stata contattata da una nota concessionaria romagnola, composta da tre sedi, 
    che vuole rinnovare il proprio sistema informatico. Non conoscendo a fondo lo specifico dominio applicativo
    si è deciso di seguire un approccio \emph{Domain Driven Design (DDD)}, quindi per prima cosa sono stati effettuati
    degli incontri con gli esperti del dominio.

  \subsection{Riunione preliminare}
  \begin{itemize}
    \item \textbf{Data}: 01/04/2023
    \item \textbf{Partecipanti}:
    \begin{itemize}
      \item 1 project manager
      \item 2 programmatori senior
      \item 1 business manager
      \item 3 esperti del dominio, sono manager esterni dell'azienda committente (uno per ogni sede)
      \item 1 tecnografo
    \end{itemize}
    \item \textbf{Resoconto}:
    \begin{itemize}
      \item \emph{Introduzione}: 
        Uno dei manager esterni ha introdotto il problema spiegando le motivazioni che hanno portato l'azienda 
          committente a voler innovare il proprio sistema informatico.
      \item \emph{Knowledge crunching}:
        É stata effettuata una prima sessione di knowledge crunching, fondamentale per l'approccio DDD, in cui è stata
          sviluppata una prima versione dell'Ubiquitous Language e dei casi d'uso.
      \item \emph{Pianificazione prossima riunione}:
        Vista la complessità del dominio applicativo e la necessità di approfondire alcuni aspetti, è stata fissata
          una seconda riunione di knowledge crunching per il giorno 05/04/2023.
    \end{itemize}
    \item \textbf{Output}:
    \begin{itemize}
      \item \emph{Registrazione della riunione}: 
        Il tecnografo, oltre ad aver trascritto i contenuti della riunione, ha anche effettuato una registrazione in caso
          ci sia bisogno di verificare alcune informazioni in futuro in modo più dettagliato. Si precisa che tutti i 
          partecipanti alla riunione, sia interni che esterni, hanno prestato il consenso.
      \item \emph{Casi d'uso}: prima versione dei casi d'uso.
      \item \emph{Ubiquitous Language}: prima, parziale, versione dell'Ubiquitous Language.
    \end{itemize}
  \end{itemize}

  \subsection{Seconda riunione di knowledge crunching}
  \begin{itemize}
    \item \textbf{Data}: 05/04/2023
    \item \textbf{Partecipanti}:
    \begin{itemize}
      \item 1 project manager
      \item 2 programmatori senior
      \item 1 business manager
      \item 3 esperti del dominio
      \item 1 tecnografo
    \end{itemize}
    \item \textbf{Resoconto}:
    \begin{itemize}
      \item \emph{Introduzione}: 
        É stato brevemente ripercorso l'ambito del progetto e quanto fatto nella riunione precedente.
      \item \emph{Knowledge crunching}:
        É stata effettuata una seconda sessione di knowledge crunching, utile per andare più nel dettaglio del dominio 
          applicativo, per chiarire alcuni dubbi emersi nella prima riunione e per rifinire i documenti prodotti nella prima riunione.
      \item \emph{Pianificazione prossima riunione}:
        Al termine della riunione si è convenuto che il dominio è stato compreso a sufficienza per poter procedere con il primo Project
        Scoping Meeting.
    \end{itemize}
    \item \textbf{Output}:
    \begin{itemize}
      \item \emph{Registrazione della riunione}
      \item \emph{Casi d'uso}: versione dettagliata dei casi d'uso.
      \item \emph{Ubiquitous Language}: versione rifinita della terminologia specifica del dominio applicativo.
    \end{itemize}
  \end{itemize}

  \subsection{Primo project scoping meeting}

  Premessa: successivamente alle riunioni di knowledge crunching i membri del team hanno redatto le
   \emph{Condition of Satisfaction (CoS)}.

  \begin{itemize}
    \item \textbf{Data}: 09/04/2023
    \item \textbf{Partecipanti}:
    \begin{itemize}
      \item 1 project manager
      \item 2 programmatori senior
      \item 1 business manager
      \item 3 esperti del dominio
      \item 1 tecnografo
    \end{itemize}
    \item \textbf{Resoconto}:
    \begin{itemize}
      \item \emph{Introduzione}: É stato effettuato un riassunto di quanto emerso dalle riunioni precedenti. 
      Inoltre, sono stati specificati i ruoli e le responsabilità dei vari membri;
      \item \emph{Descrizione del progetto}: i due manager (project e business), hanno descritto il progetto, 
        specificando gli obiettivi, le aspettative e il valore di business che questo può portare;
      \item \emph{Condition of Satisfaction (CoS)}: sono state discusse e approvate le CoS;
      \item \emph{Discussione dello stato corrente}: sono stati illustrati brevemente i documenti redatti 
        nella riunione precedente;
      \item \emph{POS e analisi dei rischi}: sono stati redatti il \emph{Project Overview Statement (POS)} 
        e l'analisi dei rischi;
      \item \emph{Pianificazione prossima riunione}:
        Si è deciso di fissare una nuova riunuone di Scoping per il giorno 12/04/2023.
    \end{itemize}
    \item \textbf{Output}:
    \begin{itemize}
      \item \emph{Versione approvata delle CoS};
      \item \emph{Project Overview Statement (POS)};
      \item \emph{Analisi dei rischi}.
    \end{itemize}
  \end{itemize}

  \subsection{Secondo project scoping meeting}
  \begin{itemize}
    \item \textbf{Data}: 12/04/2023
    \item \textbf{Partecipanti}:
    \begin{itemize}
      \item 1 project manager
      \item 2 programmatori senior
      \item 1 business manager
      \item 3 esperti del dominio
      \item 1 tecnografo
    \end{itemize}
    \item \textbf{Resoconto}:
    \begin{itemize}
      \item \emph{Introduzione}: É stato effettuato un rapido riassunto di quanto emerso nel 
        primo scoping meeting. I partecipanti sono gli stessi della scorsa riunione;
      \item \emph{Discussione dello stato corrente}: sono stati illustrati brevemente i documenti redatti 
        nella riunione precedente;
      \item \emph{SWOT analysis}: è stata effettuata la SWOT analysis;
      \item \emph{Modello di business}: si è discusso su quale modello di business fosse più adeguato per
        portare il maggior valore all'azienda;
      \item \emph{Requirements Breakdown Structure (RBS)}: sono stati raccolti tutti i requisiti in una 
        struttura gerarchica;
      \item \emph{Pianificazione prossima riunione}: si è deciso di fissare una nuova e ultima riunuone di Scoping 
        per il giorno 15/04/2023.
    \end{itemize}
    \item \textbf{Output}:
    \begin{itemize}
      \item \emph{SWOT analysis};
      \item \emph{Modello di business};
      \item \emph{RBS}.
    \end{itemize}
  \end{itemize}

  \subsection{Terzo project scoping meeting}
  \begin{itemize}
    \item \textbf{Data}: 15/04/2023
    \item \textbf{Partecipanti}:
    \begin{itemize}
      \item 1 project manager
      \item 2 programmatori senior
      \item 1 business manager
      \item 1 tecnografo
    \end{itemize}
    \item \textbf{Resoconto}:
    \begin{itemize}
      \item \emph{Introduzione}: É stato effettuato un rapido riassunto di quanto emerso nel 
        secondo scoping meeting. Vista la natura degli argomenti discussi non è stata necessaria la presenza
        degli esperti del dominio esterni;
      \item \emph{Discussione dello stato corrente}: sono stati illustrati brevemente i documenti redatti 
        nella riunione precedente;
      \item \emph{Modello PMLC}: è stato definito il PMLC che verrà utilizzato;
      \item \emph{Workflow}: è stato definito il workflow di progetto;
      \item \emph{Pianificazione prossima riunione}: La data della prossima riunione, che sarà di planning,
        verrà decisa dopo che sarà stata ricontrollata tutta la documentazione redetta nelle riunioni precedenti
        e che il progetto sarà stato definitivamente approvato da tutte le parti in causa.
    \end{itemize}
    \item \textbf{Output}:
    \begin{itemize}
      \item \emph{Modello PMCL};
      \item \emph{Workflow}.
    \end{itemize}
  \end{itemize}


  \newpage
  \section{Planning}

  \newpage
  \section{Launching/Execution}

  \newpage
  \section{Monitoring/Controlling}

  \newpage
  \section{Closing}




\end{document}
