\documentclass{article}

\usepackage[utf8]{inputenc}
\usepackage{caption}
\usepackage{subcaption}
\usepackage{natbib}
\usepackage{soul}
\usepackage{graphicx}
\usepackage{listings}
\usepackage[hyphens]{url}
\usepackage[table,xcdraw]{xcolor}
\usepackage{color}
\usepackage{textpos}
\usepackage{glossaries}
\usepackage{hyperref}
\usepackage{acronym}
\usepackage{verbatim}
\usepackage{amsmath}
\usepackage{amssymb}
\usepackage{comment}


\newcommand{\emailaddr}[1]{\href{mailto:#1}{\texttt{#1}}}

\begin{document}
\begin{titlepage}

  \newcommand{\HRule}{\rule{\linewidth}{0.5mm}}
  \center
  
  \textsc{\Large Department of Computer Science And Engineering}\\[0.5cm]
  
  \textsc{\Large University of Bologna}\\[0.6cm]
  
  \hrule width \hsize \kern 1mm \hrule width \hsize height 2pt 
  \vspace{0.8cm}
  { \large \bfseries DealerPro}\\[0.6cm]
  { \large \bfseries Descrizione dell'approccio utilizzato}\\[0.6cm]
  { \large Project Management}\\[0.6cm]
  
  
  {\bfseries{June, 2023}
  \hfill
  \bfseries{Davide Domini}}\\[0.6cm]
  
  \hrule width \hsize height 2pt \kern 1mm \hrule width \hsize height 1pt
  \vspace{0.4cm}
  %\begin{abstract}
  %\end{abstract}
  
  \end{titlepage}
  %%GLOSSARIO https://it.overleaf.com/project/646f34c78c7d27e771d0078f
  \clearpage

  \tableofcontents

  \newpage
  \section{Scoping/Initiation}

  La nostra software house è stata contattata da una nota concessionaria romagnola, composta da tre sedi, 
    che vuole rinnovare il proprio sistema informatico. Non conoscendo a fondo lo specifico dominio applicativo
    si è deciso di seguire un approccio \emph{Domain Driven Design (DDD)}, quindi per prima cosa sono stati effettuati
    degli incontri con gli esperti del dominio.

  \subsection{Riunione preliminare}
  \begin{itemize}
    \item \textbf{Data}: 01/04/2023
    \item \textbf{Partecipanti}:
    \begin{itemize}
      \item 1 project manager
      \item 2 programmatori senior
      \item 1 business manager
      \item 3 esperti del dominio, sono manager esterni dell'azienda committente (uno per ogni sede)
      \item 1 tecnografo
    \end{itemize}
    \item \textbf{Resoconto}:
    \begin{itemize}
      \item \emph{Introduzione}: 
        Uno dei manager esterni ha introdotto il problema spiegando le motivazioni che hanno portato l'azienda 
          committente a voler innovare il proprio sistema informatico;
      \item \emph{Knowledge crunching}:
        É stata effettuata una prima sessione di knowledge crunching, fondamentale per l'approccio DDD, in cui è stata
          sviluppata una prima versione dell'Ubiquitous Language e dei casi d'uso;
      \item \emph{Pianificazione prossima riunione}:
        Vista la complessità del dominio applicativo e la necessità di approfondire alcuni aspetti, è stata fissata
          una seconda riunione di knowledge crunching per il giorno 05/04/2023.
    \end{itemize}
    \item \textbf{Output}:
    \begin{itemize}
      \item \emph{Registrazione della riunione}: 
        Il tecnografo, oltre ad aver trascritto i contenuti della riunione, ha anche effettuato una registrazione in caso
          ci sia bisogno di verificare alcune informazioni in futuro in modo più dettagliato. Si precisa che tutti i 
          partecipanti alla riunione, sia interni che esterni, hanno prestato il consenso;
      \item \emph{Casi d'uso}: prima versione dei casi d'uso;
      \item \emph{Ubiquitous Language}: prima, parziale, versione dell'Ubiquitous Language.
    \end{itemize}
  \end{itemize}

  \subsection{Seconda riunione di knowledge crunching}
  \begin{itemize}
    \item \textbf{Data}: 05/04/2023
    \item \textbf{Partecipanti}:
    \begin{itemize}
      \item 1 project manager
      \item 2 programmatori senior
      \item 1 business manager
      \item 3 esperti del dominio
      \item 1 tecnografo
    \end{itemize}
    \item \textbf{Resoconto}:
    \begin{itemize}
      \item \emph{Introduzione}: 
        É stato brevemente ripercorso l'ambito del progetto e quanto fatto nella riunione precedente;
      \item \emph{Knowledge crunching}:
        É stata effettuata una seconda sessione di knowledge crunching, utile per andare più nel dettaglio del dominio 
          applicativo, per chiarire alcuni dubbi emersi nella prima riunione e per rifinire i documenti prodotti nella prima riunione;
      \item \emph{Pianificazione prossima riunione}:
        Al termine della riunione si è convenuto che il dominio è stato compreso a sufficienza per poter procedere con il primo Project
        Scoping Meeting.
    \end{itemize}
    \item \textbf{Output}:
    \begin{itemize}
      \item \emph{Registrazione della riunione}
      \item \emph{Casi d'uso}: versione dettagliata dei casi d'uso;
      \item \emph{Ubiquitous Language}: versione rifinita della terminologia specifica del dominio applicativo.
    \end{itemize}
  \end{itemize}

  \subsection{Primo project scoping meeting}

  Premessa: successivamente alle riunioni di knowledge crunching i membri del team hanno redatto le
   \emph{Condition of Satisfaction (CoS)}.

  \begin{itemize}
    \item \textbf{Data}: 09/04/2023
    \item \textbf{Partecipanti}:
    \begin{itemize}
      \item 1 project manager
      \item 2 programmatori senior
      \item 1 business manager
      \item 3 esperti del dominio
      \item 1 tecnografo
    \end{itemize}
    \item \textbf{Resoconto}:
    \begin{itemize}
      \item \emph{Introduzione}: É stato effettuato un riassunto di quanto emerso dalle riunioni precedenti. 
      Inoltre, sono stati specificati i ruoli e le responsabilità dei vari membri;
      \item \emph{Descrizione del progetto}: i due manager (project e business), hanno descritto il progetto, 
        specificando gli obiettivi, le aspettative e il valore di business che questo può portare;
      \item \emph{Condition of Satisfaction (CoS)}: sono state discusse e approvate le CoS;
      \item \emph{Discussione dello stato corrente}: sono stati illustrati brevemente i documenti redatti 
        nella riunione precedente;
      \item \emph{POS e analisi dei rischi}: sono stati redatti il \emph{Project Overview Statement (POS)} 
        e l'analisi dei rischi;
      \item \emph{Pianificazione prossima riunione}:
        Si è deciso di fissare una nuova riunuone di Scoping per il giorno 12/04/2023.
    \end{itemize}
    \item \textbf{Output}:
    \begin{itemize}
      \item \emph{Versione approvata delle CoS};
      \item \emph{Project Overview Statement (POS)};
      \item \emph{Analisi dei rischi}.
    \end{itemize}
  \end{itemize}

  \subsection{Secondo project scoping meeting}
  \begin{itemize}
    \item \textbf{Data}: 12/04/2023
    \item \textbf{Partecipanti}:
    \begin{itemize}
      \item 1 project manager
      \item 2 programmatori senior
      \item 1 business manager
      \item 3 esperti del dominio
      \item 1 tecnografo
    \end{itemize}
    \item \textbf{Resoconto}:
    \begin{itemize}
      \item \emph{Introduzione}: É stato effettuato un rapido riassunto di quanto emerso nel 
        primo scoping meeting. I partecipanti sono gli stessi della scorsa riunione;
      \item \emph{Discussione dello stato corrente}: sono stati illustrati brevemente i documenti redatti 
        nella riunione precedente;
      \item \emph{SWOT analysis}: è stata effettuata la SWOT analysis;
      \item \emph{Modello di business}: si è discusso su quale modello di business fosse più adeguato per
        portare il maggior valore all'azienda;
      \item \emph{Requirements Breakdown Structure (RBS)}: sono stati raccolti tutti i requisiti in una 
        struttura gerarchica;
      \item \emph{Pianificazione prossima riunione}: si è deciso di fissare una nuova e ultima riunuone di Scoping 
        per il giorno 15/04/2023.
    \end{itemize}
    \item \textbf{Output}:
    \begin{itemize}
      \item \emph{SWOT analysis};
      \item \emph{Modello di business};
      \item \emph{RBS}.
    \end{itemize}
  \end{itemize}

  \subsection{Terzo project scoping meeting}
  \begin{itemize}
    \item \textbf{Data}: 15/04/2023
    \item \textbf{Partecipanti}:
    \begin{itemize}
      \item 1 project manager
      \item 2 programmatori senior
      \item 1 business manager
      \item 1 tecnografo
    \end{itemize}
    \item \textbf{Resoconto}:
    \begin{itemize}
      \item \emph{Introduzione}: É stato effettuato un rapido riassunto di quanto emerso nel 
        secondo scoping meeting. Vista la natura degli argomenti discussi non è stata necessaria la presenza
        degli esperti del dominio esterni;
      \item \emph{Discussione dello stato corrente}: sono stati illustrati brevemente i documenti redatti 
        nella riunione precedente;
      \item \emph{Modello PMLC}: è stato definito il PMLC che verrà utilizzato;
      \item \emph{Workflow}: è stato definito il workflow di progetto;
      \item \emph{Pianificazione prossima riunione}: La data della prossima riunione, che sarà di planning,
        verrà decisa dopo che sarà stata ricontrollata tutta la documentazione redetta nelle riunioni precedenti
        e che il progetto sarà stato definitivamente approvato da tutte le parti in causa.
    \end{itemize}
    \item \textbf{Output}:
    \begin{itemize}
      \item \emph{Modello PMCL};
      \item \emph{Workflow}.
    \end{itemize}
  \end{itemize}


  \newpage
  \section{Planning}

  Per la fase di planning, essendo questo un progetto medio-grande, si è prevista una durata massima di 
    \textbf{tre giorni}. Potrà essere fatta un'eccezione per l'aggiunta di \textbf{un giorno} nel caso 
    in cui si riscontrino particolari difficoltà.

  \subsection{Prima joint project planning session}
  \begin{itemize}
    \item \textbf{Data}: 20/04/2023
    \item \textbf{Partecipanti}: 
    \begin{itemize}
      \item 1 project manager
      \item 2 programmatori senior
      \item 1 programmatore junior
      \item 1 tecnografo
    \end{itemize}
    \item \textbf{Facilities}:
    \begin{itemize}
      \item Proiettore e computer
      \item Sala riunioni privata
      \item Lavagna e pennarelli
    \end{itemize}
    \item \textbf{Resoconto}:
    \begin{itemize}
      \item \emph{Introduzione}:
        É stato effettuato un riassunto degli aspetti emersi nella fase di scoping.
      \item \emph{Microservizi}: 
        Sono stati definiti i vari microservizi che comporranno l'applicativo. 
        Questi sono:
        \begin{itemize}
          \item \textbf{Auth service}: servizio che permette di gestire 
            l’autenticazione e l’autorizzazione degli utenti;
          \item \textbf{Dealers service}: servizio che permette di gestire
            preventivi e ordini;
          \item \textbf{Management service}: servizio che permette di ottenere statistiche e
            report utili per prendere decisioni strategiche;
          \item \textbf{Customers service}: servizio che permette di gestire le esigenze 
            dei clienti che hanno già acquistato un’auto, come ad esempio: 
            gestione degli appuntamenti ed avvisi di manutenzione obbligatoria;
          \item \textbf{Employees service}: servizio che permette di gestire le 
            ferie e i permessi dei dipendenti;
          \item \textbf{Client frontend}: interfaccia che permette agli utenti 
            di interagire con l’applicativo.
        \end{itemize}
        \item \emph{Core domain chart}: è stato creato il core domain chart utile per classificare i microservizi 
          descritti in precedenza;
        \item \emph{Context map}: è stata creata la context map che definisce le relazioni e le dipendenze fra i 
          vari microservizi;
        \item \emph{Approccio di progetto}: è stato identificato il miglior approccio per portare avanti il
          progetto, questo è il \textbf{Evolutionary development waterfall};
        \item \emph{Pianificazione prossima riunione}: si è deciso di fissare la prossima riunione di planning
          per il giorno 21/04/2023.
    \end{itemize}
    \item \textbf{Output}:
    \begin{itemize}
      \item \emph{Approccio di progetto};
      \item \emph{Core domain chart};
      \item \emph{Context map}.
    \end{itemize}
  \end{itemize}

  \subsection{Seconda joint project planning session}
  \begin{itemize}
    \item \textbf{Data}: 21/04/2023
    \item \textbf{Partecipanti}:
    \begin{itemize}
      \item 1 project manager
      \item 2 programmatori senior
      \item 1 programmatore junior
      \item 1 tecnografo
    \end{itemize}
    \item \textbf{Facilities}:
    \begin{itemize}
      \item Proiettore e computer
      \item Sala riunioni privata
      \item Post-it colorati
      \item Lavagna e pennarelli
    \end{itemize}
    \item \textbf{Resoconto}:
    \begin{itemize}
      \item \emph{Introduzione};
        É stato effettuato un riassunto della prima riunione di planning;
      \item \emph{Work Breakdown Structure}:
        A partire dalla Requirements Breakdown Structure è stata definita la Work Breakdown Structure. ;
        A partire dalla WBS, sono stati identificati i task e riportati su dei post-it colorati (un colore per ogni sottoprogetto);
      \item \emph{Project network diagram}:
        Partendo dai post-it identificati alla fase precedente sono state identificate le relazioni fra i 
        vari task per create il project network diagram.
      \item \emph{Pianificazione prossima riunione}: si è deciso di fissare la terza ed ultima riunione 
        di planning per il giorno 22/04/2023.
    \end{itemize}
    \item \textbf{Output}: 
    \begin{itemize}
      \item \emph{Work Breakdown Structure};
      \item \emph{Project network diagram}.
    \end{itemize}
  \end{itemize}

  \subsection{Terza joint project planning session}
  \begin{itemize}
    \item \textbf{Data}: 22/04/2023
    \item \textbf{Partecipanti}:
    \begin{itemize}
      \item 1 project manager
      \item 2 programmatori senior
      \item 1 programmatore junior
      \item 1 tecnografo
    \end{itemize}
    \item \textbf{Facilities}:
    \begin{itemize}
      \item Proiettore e computer
      \item Sala riunioni privata
      \item Post-it colorati
      \item Lavagna e pennarelli
    \end{itemize}
    \item \textbf{Resoconto}:
    \begin{itemize}
      \item \emph{Introduzione};
        É stato effettuato un riassunto della seconda riunione di planning;
      \item \emph{Stima durata attività}: 
        É stata stimata la durata per ogni attività presente nel WBS (aggiornando anche il 
        project network diagram per aggiungere questa informazione);
      \item \emph{Stima risorse}: 
        Sono state definite le risorse necessarie per portare a termine ogni sottoprogetto;
      \item \emph{Stima tempi}:
        Tenendo in considerazione la stima della durata delle attività e delle risorse definite
        in precedenza si è stimata la durata del progetto in termini di sprint.
    \end{itemize}
    \item \textbf{Output}: 
    \begin{itemize}
      \item \emph{Stima delle risorse};
      \item \emph{Stima dei tempi};
      \item \emph{Project network diagram aggiornata}.
    \end{itemize}
  \end{itemize}

  \subsection{Project proposal}
  È stata scritta una proposta di progetto ed è stata sottoposta al processo di approvazione da parte del Senior Manager. 
  È possibile consultare tale documento, il quale è denominato  \textcolor{teal}{DealerPro Project Proposal}.
  Il documento e i suoi allegati sono stati valutati. È stata quindi fornita l’approvazione a procedere alla fase
  di Launching.


  \newpage
  \section{Launching/Execution}

  \subsection{Recruiting/staffing}

  \begin{itemize}
    \item \textbf{Core team}: team composto da 2 programmatori senior;
    \item \textbf{Project manager}: ruolo ricoperto dallo stesso manager che ha presenziato 
      durante le riunioni di scoping e planning;
    \item \textbf{Co-project manager}: membro del core team che affianca il project manager;
    \item \textbf{Developer team}: 2 sviluppatori junior;
    \item \textbf{Client team}: membri identificati dal cliente, questi forniscono feedback sui 
      risultati ottenuti ad ogni sprint in modo da indenficare quanto prima eventuali problemi;
    \item \textbf{Contracted team}: libero professionista a contratto che si è occupato di 
      sviluppare l'interfaccia grafica.
  \end{itemize}

  \subsection{Kick-off meeting}
  \begin{itemize}
    \item \textbf{Data}: 2/05/2023
    \item \textbf{Partecipanti}: 
    \begin{itemize}
      \item 1 project manager
      \item 2 programmatori senior
      \item 2 programmatori junior
      \item 1 tecnografo
      \item 1 UI designer
      \item Client team
    \end{itemize}
    \item \textbf{Facilities}:
    \begin{itemize}
      \item Proiettore e computer
      \item Sala riunioni privata
      \item Lavagna e pennarelli
    \end{itemize}
    \item \textbf{Resoconto}:
    \begin{itemize}
      \item \emph{Introduzione}:
        É stato presentato, dal project manager, il progetto specificando lo scopo, il business
          value e la soluzione che si intende sviluppare;
      \item \emph{Presentazione dei membri del team di sviluppo}: 
       Ogni membro del team di sviluppo è stato presentando, specificando per ognuno il suo ruolo
        e le sue responsabilità all'interno del progetto;
      \item \emph{Presentazione documentazione sviluppata fino ad ora}:
        Il project manager ha presentato la documentazione che è stata redatta nelle fasi precedenti;
      \item \emph{Regole operative del team}:
        Sono state discusse le regole operative che il team deve seguire durante lo svolgimento del progetto;
      \item \emph{Piano per la qualità}:
        È stato discusso il piano per la qualità che il team deve seguire durante lo svolgimento del progetto;
      \item \emph{Piano di progetto}:
        Sono state discusse le disponibilità dei vari membri del team per integrarle con l'agenda del progetto.
    \end{itemize}
    \item \textbf{Output}:
    \begin{itemize}
      \item \emph{Regole operative per il team};
      \item \emph{Piano per la qualità}.
    \end{itemize}
  \end{itemize}

  \subsubsection{Regole operative per il team}

  \paragraph{Riunioni}

  \begin{itemize}
    \item \textbf{Project review meeting}: riunione condotta all'inzio di ogni settimana. Durante questa riunione vengono 
      discussi i risultati dello sprint precedente e vengono pianificate le attività da svolgere durante lo sprint corrente;
    \item \textbf{15 minutes daily status meeting}: riunione condotta ogni giorno ad inizio mattinata. Durante questa riunione  
      ogni membro del team discute brevemente il lavoro svolto il giorno prima, quello che prevede di svolgere il giorno corrente e
      se è presente qualche difficoltà nel raggiungere gli obbiettivi;
    \item \textbf{Problem resolution meeting}: riunioni che vengono condotte solo in caso di evenienza, sono volte a risolvere un particolare
      problema che si è presentato durante lo svolgimento del progetto. A queste riunioni partecipano solo i membri che sono coinvolti
      direttamente nel problema. Solo in casi estremi, in cui non si riesce in nessun modo a trovare una soluzione potranno essere convocati
      anche gli altri membri.
  \end{itemize}

  \paragraph{Modalità di comunicazione}
  \begin{itemize}
    \item \textbf{Task svolti}: per la gestione dei task verrà usato uno specifico software -- \emph{Trello} --
      che permette di gestire una board dove classificare un task in base al suo stato e di notificare in automatico
      gli altri membri del team in caso di cambiamento di stato. Per esempio, una volta finito un task questo dovrà 
      essere spostato nella colonna \emph{Done};
    \item \textbf{Comunicazioni generiche}: per qualsiasi altra comunicazione riguardante il progetto, invece, verrà
      utilizzata un'apposita applicazione di messagistica istantanea -- \emph{Slack} -- che permette di creare canali 
      dedicati in base al topic.
  \end{itemize}


  \subsubsection{Piano per la qualità}
  Si è deciso di adottare le seguenti Good Practices per raggiungere l'obbiettivo dell'eccellenza tecnica:
  \begin{itemize}
    \item Utilizzo di un repository per ogni sottoprogetto;
    \item Utilizzo di GitFlow all'interno di ogni repository;
    \item Utilizzo di Conventional Commits per la scrittura dei commit;
    \item Commit scritti rigorosamente in inglese;
    \item Utilizzo di firma dei commit con chiave GPG;
    \item Utilizzo di un sistema di CI/CD per l'automazione dei test e del deploy;
    \item Utilizzo di un build tool per la risoluzione delle dipendenze date da librerie esterne;
    \item Utilizzo di sistema automatico per la gestione dell'update delle versioni delle librerie esterne;
    \item Inclusione della documentazione in ogni repository;
    \item Scrittura di commenti all’interno del codice seguendo le convenzioni del linguaggio;
    \item Utilizzo di pattern di programmazione consolidati per garantire riusabilità, scalabilità ed estensibilità
      del codice;
    \item Controllo automatico della compatibilità di tutte le licenze utilizzate;
    \item Conteinerizzazione del codice;
    \item Adozione di un linter automatico per la formattazione del codice seguendo le convenzioni del
      linguaggio;
    \item Utilizzo di un tool per l’identificazione di codice ripetuto all’interno dello stesso progetto.
  \end{itemize}


  \newpage
  \section{Monitoring/Controlling}

  \subsection{Sprint review}
  Ogni due settimane, quindi ogni due sprint, viene organizzata una riunione con l'azienda cliente per mostrare i progressi.
  In questo modo si hanno feedback continui e si può correggere eventuali problemi in modo tempestivo. Le riunioni sono strutturate 
  secondo la seguente agenda:
  \begin{itemize}
    \item \emph{Introduzione}: viene descritto il lavoro fatto nelle ultime due settimane, ponendo particolare enfasi sul valore
      aggiunto dalle nuove feature;
    \item \emph{Demo del sistema}: viene mostrato brevemente il sistema in funzione e come l'utente può interagire con esso;
    \item \emph{Discussione}: l'azienda cliente può fare domande e richiedere chiarimenti sulle nuove funzionalità, inoltre esprime 
      i feedback su quanto visto.
  \end{itemize} 

  \subsection{Project status meeting}

  \subsubsection*{15 Minutes Daily Status Meeting}
  In questa riunione, giornaliera, ogni membro riporta brevemente lo stato del proprio lavoro.
  In particolare, lo stato può essere:
  \begin{itemize}
    \item \emph{In schedula};
    \item \emph{In anticipo rispetto alla schedula}, qui sarà necessario fare una stima di quanto;
    \item \emph{In ritardo rispetto alla schedula}, qui sarà necessario specificare
      di quanto e se c'è il bisogno di un aiuto.
  \end{itemize}

  \subsubsection*{Problem Resolution Meeting}
  Nel caso in cui durante uno dei vari meeting emerga un problema per il quale è necessario l'intervento di più persone, questo viene aggiunto
    nell' \emph{issue log} e viene pianificata un'apposita riunione per risolverlo. Queste riunioni prendono il nome 
    di \emph{Problem Resolution Meeting} e vi partecipano solo i membri direttamente coinvolti nel problema. Va concordata la definizione 
    del problema e chi è il proprietario del problema. Si effettua del Brainstorming per cercare di trovare la soluzione del problema, si assegna
    una priorità alle soluzioni individuate e si aggiorna l’Issues Log. Solo se necessario si pianifica il prossimo
    meeting.
  

  \subsection{Problem escalation strategy}

 In caso insorga qualche problema la figura da interpellare è il \emph{Project Manager}. 
    Sarà suo compito esaminare le relazioni di dipendenza, riorganizzare la schedula e riassegnare le risorse in caso sia necessario.

  \subsection{Gestione report di controllo}

  \subsubsection*{Trello}
  \emph{Trello} è uno strumento software utilizzato per la pianificazione del lavoro durante gli sprint, questo strumento permette
    anche di associarea ad ogni task una checklist, in questo modo si può decomporre ulteriormente un task e capire in maniera rapida
    lo stato di avanzamento. Inoltre, è stato definito un sistema di etichette che vengono associate ad ogni task per definirne il suo stato,
    queste sono: \emph{done}, \emph{documentato}, \emph{testato}, \emph{problema riscontrato}.
    Infine, viene anche descritto lo stato di avanzamento del task utilizzando una tabella le cui colonne sono: 
    \emph{Todo}, \emph{In progress}, \emph{Done}.


  \subsubsection*{Cumulative Reports}
  Durante la fase di Retrospettiva di ogni sprint viene redatto un \emph{Cumulative Report}.
    Questo documento è un grafico formato da due assi del piano cartesiano: sulle x c’è lo scorrere del tempo
    (scala in settimane) mentre sulle y viene rappresentato il ritardo totale delle attività di progetto (scala in ore).


  \newpage
  \section{Closing}

  \subsection{Riunione post implementazione}
  \begin{itemize}
    \item \textbf{Data}: 10/07/2023
    \item \textbf{Partecipanti}: 
    \begin{itemize}
      \item 1 project manager
      \item 2 programmatori senior
      \item 2 programmatori junior
      \item 1 tecnografo
    \end{itemize}
    \item \textbf{Facilities}:
    \begin{itemize}
      \item Proiettore e computer
      \item Sala riunioni privata
      \item Lavagna e pennarelli
    \end{itemize}
    \item \textbf{Resoconto}:
    \begin{itemize}
      \item \emph{Obbiettivi}: 
        il sistema risulta funzionante nella sua interezza e gli obiettivi del progetto sono stati raggiunti;
      \item \emph{Risorse}:
        le stime effettuate sono risultate corrette, permettendo la riuscita del progetto nei tempi e costi indicati;
      \item \emph{Specifiche}:
        tutte le specifiche sono state rispettate;
      \item \emph{Deliverables}:
        i deliverables di progetto riescono a raggiungere il loro scopo e sono stati testati in modo da
        garantirne la correttezza, inoltre, anche l'azienda committente ha confermato il corretto 
        funzionamento del sistema;
      \item \emph{Business value}:
        l'azienda committente ha confermato il valore apportato dal nuovo software;
      \item \emph{Metodologia di gestione del progetto}:
        inizialmente i programmatori junior hanno avuto qualche difficoltà ad adattarsi alla metodologia scelta
        e ai vincoli di qualità ed eccellenza tecnica richiesti, queste sono state superate grazie all'aiuto
        dei programmatori senior e del project manager.
    \end{itemize}
  \end{itemize}


  \subsection{Installazione del software}

  Il deploy in cloud del software ha seguito un approccio phased durante i vari sprint. 
    Il cliente ha poi deciso di iniziare ad usare il software con un approccio cut-over, quindi 
    nel momento in cui il progetto è stato terminato è stato scelto un giorno per spegnere il vecchio sistema
    e passare all'utilizzo del nuovo.



\end{document}
