\documentclass{article}

\usepackage[utf8]{inputenc}
\usepackage{caption}
\usepackage{subcaption}
\usepackage{natbib}
\usepackage{soul}
\usepackage{graphicx}
\usepackage{listings}
\usepackage[hyphens]{url}
\usepackage[table,xcdraw]{xcolor}
\usepackage{color}
\usepackage{textpos}
\usepackage{glossaries}
\usepackage{hyperref}
\usepackage{acronym}
\usepackage{verbatim}
\usepackage{amsmath}
\usepackage{amssymb}
\usepackage{comment}
\usepackage{xcolor}


\newcommand{\emailaddr}[1]{\href{mailto:#1}{\texttt{#1}}}
\renewcommand{\labelenumii}{\arabic{enumi}.\arabic{enumii}}
\renewcommand{\labelenumiii}{\arabic{enumi}.\arabic{enumii}.\arabic{enumiii}}
\renewcommand{\labelenumiv}{\arabic{enumi}.\arabic{enumii}.\arabic{enumiii}.\arabic{enumiv}}

\begin{document}
\begin{titlepage}

  \newcommand{\HRule}{\rule{\linewidth}{0.5mm}}
  \center
  
  \textsc{\Large Department of Computer Science And Engineering}\\[0.5cm]
  
  \textsc{\Large University of Bologna}\\[0.6cm]
  
  \hrule width \hsize \kern 1mm \hrule width \hsize height 2pt 
  \vspace{0.8cm}
  { \large \bfseries DealerPro}\\[0.6cm]
  { \large \bfseries Project Overview Statement}\\[0.6cm]
  { \large Project Management}\\[0.6cm]
  
  
  {\bfseries{June, 2023}
  \hfill
  \bfseries{Davide Domini}}\\[0.6cm]
  
  \hrule width \hsize height 2pt \kern 1mm \hrule width \hsize height 1pt
  \vspace{0.4cm}
  
  \end{titlepage}

  \clearpage
  
  \section*{Problema/Opportunità}
  La gestione di una concessionaria di auto comporta una serie di processi ricorrenti e standardizzati che 
    vengono svolti dai dipendenti. Una nota concessionaria romagnola, composta da tre sedi, ha deciso di 
    rinnovare il proprio sistema informatico in quanto ritenuto ormai obsoleto.

  Il sistema attuale è stato sviluppato negli anni, in maniera frammentata, da diversi team di sviluppatori.
    Oltretutto, le attuali sedi che compongono l'azienda, in principio, erano indipendenti e sono state acquisite
    nel tempo. Questo aspetto ha portato il sistema informatico ad essere ancor più disomogeneo rendendo difficile 
    e tediosa l'integrazione dei dati delle diverse sedi.

  Tra i principali limiti del sistema attuale dunque troviamo:
  \begin{itemize}
    \item Scarsa capacità di fornire ai manager informazioni utili e aggregate per prendere decisioni strategiche;
    \item Lentezza e limitata user-experience per i dipendenti, questo causa in loro frustrazione e scarsa produttività. 
    \item Spesso i dipendenti inseriscono solo informazioni parziali o non aggiornate, questo causa una scarsa qualità dei dati;
    \item I dipendenti sono costretti a svolgere attività ripetitive e manuali che potrebbero essere automatizzate;
    \item Scarsa organizzazione interna (ad esempio, ferie e permessi dei dipendenti);
    \item Manca totalmente un sistema di gestione dei clienti per quanto riguarda gli appuntamenti periodici 
    (ad esempio, per il tagliando).
  \end{itemize}

  Le opportunità di business di questo progetto sono presenti sia per il committente che per il team di sviluppo. Il cliente potrà
    rinnovare e automatizzare i processi interni, aumentare la soddisfazione dei dipendenti e dei clienti, avere un
    monitoring più pervasivo dei risultati aziendali facilitando il lavoro dei manager. 
    Il team di svilupp, invece, potrà acquisire nuove competenze e conoscenze su un nuovo dominio applicativo che potrebbe portare a vendere
    un servizio molto simile anche ad altre aziende del settore.


  \section*{Obbiettivi del progetto}

  L'obbiettivo di questo progetto è realizzare un sistema software volto ad automatizzare e migliorare 
    i processi interni di una concessionaria di auto.
    Il principio chiave di DealerPro è la semplicità di utilizzo, questa permetterà di aumentare la 
    soddisfazione e l'efficienza dei dipendenti andando, di conseguenza, a migliorare anche la qualità
    del servizio offerto ai clienti della consessionaria.

  La soluzione finale adotterà un modello a microservizi collocati in cloud, questo permetterà alla
    consessionaria di non dover gestire un'infrastruttura interna, inoltre essendo un dominio altamente
    standardizzato sotto molti punti di vista sarà possibile vendere l'applicativo anche ad altre 
    consessionarie riducendo il prezzo di vendita alla singola azienda.
 
  \section*{Importanza degli obbiettivi}
  Per ogni obiettivo viene indicata una percentuale, la quale rappresenta l’importanza associata in
    termini di Scope e Budget.
    \begin{itemize}
        \item \textbf{[15\%]} Sviluppo del client mediante il quale gli utenti potranno interagire con il sistema;
        \item \textbf{[10\%]} Sviluppo del sistema di autenticazione;
        \item \textbf{[10\%]} Sviluppo del sistema di gestione dei dipendenti;
        \item \textbf{[10\%]} Sviluppo del sistema di gestione dei clienti;
        \item \textbf{[15\%]} Sviluppo del sistema di gestione delle vendite;
        \item \textbf{[25\%]} Sviluppo del sistema di monitoring per i manager;
        \item \textbf{[10\%]} Documentazione dei vari microservizi;
        \item \textbf{[5\%]} Manuale utente.
    \end{itemize}
  \section*{Criteri di successo}

  \begin{itemize}
    \item Aumento delle entrate derivanti da una maggiore organizzazione ed efficienza dei processi aziendali
        (ad esempio, gli appuntamenti);
    \item Costi della manutenzione del sistema ridotti in quanto non è più necessario avere 
        un'infrastruttura interna;
    \item Miglioramento del servizio offerto ai clienti.
  \end{itemize}




  \section*{Premesse}
  \begin{itemize}
    \item La concessionaria è dotata di una connessione ad internet e di dispositivi hardware adeguati;
    \item Si disporrà di un team sufficientemente numeroso e competente per poter realizzare il progetto;
    \item Sarà utilizzata un'infrastruttura Cloud;
    \item Sarà garantita la presenza di esperti del dominio;
    \item Essendo il progetto commissionato da un cliente, le deadline saranno stringenti e potranno 
        essere fatte solo rare eccezioni;
    \item Verranno individuati nuovi potenziali acquirenti.
  \end{itemize}

  \section*{Rischi}
  L'analisi dei rischi è presente nel documento \textcolor{teal}{DealerPro Risk Analysis}.

  \section*{Ostacoli}
  \begin{itemize}
    \item Non saranno individuate altre aziende disposte a comprare il servizio;
    \item Il tempo di learning dei dipendenti sarà troppo lungo.
  \end{itemize}

 


\end{document}
