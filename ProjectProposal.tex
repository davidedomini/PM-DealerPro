\documentclass{article}

\usepackage[utf8]{inputenc}
\usepackage{caption}
\usepackage{subcaption}
\usepackage{natbib}
\usepackage{soul}
\usepackage{graphicx}
\usepackage{listings}
\usepackage[hyphens]{url}
\usepackage[table,xcdraw]{xcolor}
\usepackage{color}
\usepackage{textpos}
\usepackage{glossaries}
\usepackage{hyperref}
\usepackage{acronym}
\usepackage{verbatim}
\usepackage{amsmath}
\usepackage{amssymb}
\usepackage{comment}


\newcommand{\emailaddr}[1]{\href{mailto:#1}{\texttt{#1}}}

\begin{document}
\begin{titlepage}

  \newcommand{\HRule}{\rule{\linewidth}{0.5mm}}
  \center
  
  \textsc{\Large Department of Computer Science And Engineering}\\[0.5cm]
  
  \textsc{\Large University of Bologna}\\[0.6cm]
  
  \hrule width \hsize \kern 1mm \hrule width \hsize height 2pt 
  \vspace{0.8cm}
  { \large \bfseries DealerPro}\\[0.6cm]
  { \large \bfseries Project Proposal}\\[0.6cm]
  { \large Project Management}\\[0.6cm]
  
  
  {\bfseries{June, 2023}
  \hfill
  \bfseries{Davide Domini}}\\[0.6cm]
  
  \hrule width \hsize height 2pt \kern 1mm \hrule width \hsize height 1pt
  \vspace{0.4cm}
  
  \end{titlepage}

  \clearpage

  \tableofcontents

  \newpage
  
  \section{Executive Summary}

  \subsection{Problema}

  La gestione di una concessionaria di auto comporta una serie di processi ricorrenti e standardizzati che 
    vengono svolti dai dipendenti. Una nota concessionaria romagnola, composta da tre sedi, ha deciso di 
    rinnovare il proprio sistema informatico in quanto ritenuto ormai obsoleto.

  Il sistema attuale è stato sviluppato negli anni, in maniera frammentata, da diversi team di sviluppatori.
    Oltretutto, le attuali sedi che compongono l'azienda, in principio, erano indipendenti e sono state acquisite
    nel tempo. Questo aspetto ha portato il sistema informatico ad essere ancor più disomogeneo rendendo difficile 
    e tediosa l'integrazione dei dati delle diverse sedi.

  Tra i principali limiti del sistema attuale dunque troviamo:
  \begin{itemize}
    \item Scarsa capacità di fornire ai manager informazioni utili e aggregate per prendere decisioni strategiche;
    \item Lentezza e limitata user-experience per i dipendenti, questo causa in loro frustrazione e scarsa produttività. 
    \item Spesso i dipendenti inseriscono solo informazioni parziali o non aggiornate, questo causa una scarsa qualità dei dati;
    \item I dipendenti sono costretti a svolgere attività ripetitive e manuali che potrebbero essere automatizzate;
    \item Scarsa organizzazione interna (ad esempio, ferie e permessi dei dipendenti);
    \item Manca totalmente un sistema di gestione dei clienti per quanto riguarda gli appuntamenti periodici 
    (ad esempio, per il tagliando).
  \end{itemize}

  \subsection{Soluzione}

  DealerPro è un servizio software volto a semplificare ed automatizzare i processi della concessionaria di auto.
    Il servizio è composto da diversi moduli che si integrano tra loro, ovvero:
    \begin{itemize}
      \item \textbf{Venditori}: permette di gestire preventivi e ordini;
      \item \textbf{Clienti}: permette di gestire le esigenze dei clienti che hanno già acquistato un'auto, 
      come ad esempio: gestione degli appuntamenti ed avvisi di manutenzione obbligatoria;
      \item \textbf{Dipendenti}: permette di gestire le ferie e i permessi dei dipendenti;
      \item \textbf{Management}: permette ai manager di ottenere statistiche e report utili 
        per prendere decisioni strategiche.
    \end{itemize}

  \subsection{Impatto}

  


  \subsection{Business}

  \newpage
  \section{Background} %makes sense?

  \newpage
  \section{Obiettivi}

  \newpage
  \section{Overview dell'approccio}

  \newpage
  \section{Riassunto di tempi e risorse}




\end{document}
