\documentclass{article}

\usepackage[utf8]{inputenc}
\usepackage{caption}
\usepackage{subcaption}
\usepackage{natbib}
\usepackage{soul}
\usepackage{graphicx}
\usepackage{listings}
\usepackage[hyphens]{url}
\usepackage[table,xcdraw]{xcolor}
\usepackage{color}
\usepackage{textpos}
\usepackage{glossaries}
\usepackage{hyperref}
\usepackage{acronym}
\usepackage{verbatim}
\usepackage{amsmath}
\usepackage{amssymb}
\usepackage{comment}
\usepackage{xcolor}


\newcommand{\emailaddr}[1]{\href{mailto:#1}{\texttt{#1}}}
\renewcommand{\labelenumii}{\arabic{enumi}.\arabic{enumii}}
\renewcommand{\labelenumiii}{\arabic{enumi}.\arabic{enumii}.\arabic{enumiii}}
\renewcommand{\labelenumiv}{\arabic{enumi}.\arabic{enumii}.\arabic{enumiii}.\arabic{enumiv}}

\begin{document}
\begin{titlepage}

  \newcommand{\HRule}{\rule{\linewidth}{0.5mm}}
  \center
  
  \textsc{\Large Department of Computer Science And Engineering}\\[0.5cm]
  
  \textsc{\Large University of Bologna}\\[0.6cm]
  
  \hrule width \hsize \kern 1mm \hrule width \hsize height 2pt 
  \vspace{0.8cm}
  { \large \bfseries DealerPro}\\[0.6cm]
  { \large \bfseries Criteri di accettazione}\\[0.6cm]
  { \large Project Management}\\[0.6cm]
  
  
  {\bfseries{June, 2023}
  \hfill
  \bfseries{Davide Domini}}\\[0.6cm]
  
  \hrule width \hsize height 2pt \kern 1mm \hrule width \hsize height 1pt
  \vspace{0.4cm}
  
  \end{titlepage}

  \clearpage
  
  \section*{Criteri di accettazione}
  
  \begin{itemize}
    \item \textbf{\emph{Affidabilità}}: durante l'orario di lavoro, il sistema deve dimostrare 
        stabilità e affidabilità, mantenendo tempi di indisponibilità o 
        interruzioni di servizio minimi e garantendo una disponibilità superiore al 99\%.
    \item \textbf{\emph{Facilità d'uso}}: è necessario che il sistema sia intuitivo e facile 
        da utilizzare, presentando un'interfaccia utente chiara e ben strutturata, 
        al fine di ridurre al minimo la necessità di formazione specifica.
    \item \textbf{\emph{Sicurezza dei dati}}: affinché sia conforme alle normative sulla privacy 
        e alla sicurezza delle informazioni, il sistema deve assicurare la protezione e 
        la riservatezza dei dati sensibili attraverso l'implementazione di meccanismi di 
        crittografia e controlli di accesso adeguati.
    \item \textbf{\emph{Scalabilità}}: il sistema deve essere in grado di gestire una grande quantità
        di dati senza che questo vada ad influenzare negativamente la stabilità e/o le prestazioni.
    \item \textbf{\emph{Manutenibilità}}: il sistema deve essere facilmente manutenibile, 
        in modo da poter essere modificato e aggiornato.
    \item \textbf{\emph{Efficienza}}: per le operazioni più comuni, il sistema deve essere in grado 
        di elaborare le informazioni rapidamente e tempestivamente, assicurando un tempo 
        di risposta medio inferiore a 3 secondi.
    \item \textbf{\emph{Supporto tecnico}}: il fornitore del sistema deve garantire un supporto tecnico 
        completo, con tempi di risposta rapidi per le richieste di assistenza e un solido supporto 
        post-vendita per affrontare eventuali problemi e bug.
    \item \textbf{\emph{Accessibilità}}: al fine di garantire l'accesso efficace del software a tutti
        gli utenti, compresi quelli con disabilità visive, uditive o motorie, il sistema deve essere
        progettato tenendo conto dei principi di accessibilità, e deve essere conforme 
        alle direttive AGID.
    \item \textbf{\emph{Integrazione}}: i vari microservizi che compongono il sistema devono integrarsi
        senza problemi fra loro consentendo uno scambio di dati efficiente e sicuro.
  \end{itemize}

 


\end{document}
